\section*{Abstract}
\addcontentsline{toc}{section}{Abstract}

Transport modelling initially relied on static models, which remain widely used in practice despite their strong assumption that congestion levels remain constant during peak and non-peak hours.
This assumption was challenged by several economists
in the second half of the last century.

The main alternative approach was formalized by \textcite{de1983stochastic},
defining what is known as the \(\alpha\)-\(\beta\)-\(\gamma\) model,
a stochastic departure framework.
Here the commuters make trade-offs between congestion levels and early arrivals relative to an ideal arrival time, denoted as \(t^*\).
According to the \(\alpha\)-\(\beta\)-\(\gamma\) model,
\(\alpha\) is the value of time, \(\beta\) the unit cost of early arrival and \(\gamma\) the unit cost of late arrival.
This theoretical model assumes congestion to be endogenous,
and studies the situation in which it is stabilized at equilibrium.

The equilibrium solutions, as can be imagined,
strictly depend on the values of the parameters \(\alpha\), \(\beta\) and \(\gamma\).
However, estimation of them is impractical, as it relies uniquely on costly and infrequent surveys.
Moreover, theoretical results are often impractical, as even in a steady state, actual travel times can vary significantly.

The aim of this thesis is therefore to develop a methodology for calibrating the values of the parameters \(\alpha, \beta, \gamma\) and \(t^*\),
for a group of user travelling with exogenous congestion pattern.
Being the model applicable to an arbitrary network, the congestion is not assumed to be the solution of a classical equilibrium model:
we assume it to be known,
either from simulations or direct observation.
We thus derive the conditions corresponding to the minimization of the individual cost,
for each commuter.
Based on this, we are able to describe the distribution of parameters for a group of commuters.

Our method shows great accuracy when evaluated on a synthetic dataset.



%%% Local Variables:
%%% mode: LaTeX
%%% TeX-master: "../main"
%%% End:
