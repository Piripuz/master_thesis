\chapter{Results}
\label{chap:res}

In this chapter, the results of the research will be presented.

The first step for building the model is some analytical work:
in section MINIMIZING where the cost function is minimized will be studied,
and a characterization of the minima will be done.
This characterization will then be used in section LIKELIHOOD,
where the statistical framework will be developed and the likelihood function introduced.

Once an expression for the likelihood function is found,
it can be numerically implemented:
details of the implementation, and its results on the various test types,
are shown in the following sections.

\section{Finding the Minimum of the Cost Function}
\label{sec:minimum}

For characterizing where the cost function is minimized,
it is a good idea to simplify it as much as possible.
Some theoretical preliminaries are hence necessary.

\subsection{Theoretical Preliminaries}
\label{sec:pre_minimizing}

Recall that, from equation \eqref{eq:cost_inf}, we are assuming the cost to be

\begin{equation*}
  C(t; \alpha, \beta, \gamma, t^*) = \alpha tt(t) + \beta [t^* - tt(t) - t]^+ + \gamma[tt(t) + t - t^*]^+
\end{equation*}

where

\begin{itemize}
\item \(t^*\) is the desired arrival time
\item \(\alpha\) is the value of time spent travelling
\item \(\beta\) is the value of time spent waiting at the destination
\item \(\gamma\) is the value of time arriving late at the destination
\item \(tt(t_d)\) is the time spent travelling if departing at time \(t_d\)
\item \([x]^+ = \max(0, x)\)
\end{itemize}

The first observation is that minimizing a function is equivalent to minimize a scaled version of it:
instead of minimizing the function \(C(t_d)\),
I will study the minimization problem on the function \(C(t_d)/\alpha\) or,
equivalently, normalize the parameter \(\alpha\) to 1.

From now on a slightly different cost function will be considered:
\begin{equation}
  \label{eq:cost_no_alpha}
  C(t; \beta, \gamma, t^*) = tt(t) + \beta [t^* - tt(t) - t]^+ + \gamma[tt(t) + t - t^*]^+
\end{equation}
where the numerical values of the parameters \(\beta, \gamma\) are actually representing the values for \(\beta/\alpha, \gamma/\alpha\).

Note now that most of the cost function depend on when the user \textit{decides to arrive} rather than when the user decides to leave.
Expressing it in terms of the arrival time may thus be more natural,
and simplify the expression of the cost function.

For doing this, define the \textit{arrival time} \(t_a\), in function of the departure time \(t_d\):
\[t_a(t_d) = t_d + tt(t_d)\]

Suppose now that there exists a function \(tt_a(t)\) expressing the travel time in function of the arrival time:
\begin{equation}
  \label{eq:tt_a_def}
  tt_a(t_a(t_d)) = tt(t_d)
\end{equation}

In this case, the expression \eqref{eq:cost_no_alpha} considerably simplifies:
\begin{equation}
  \label{eq:cost_t_d_t_a}
  \begin{split}
    C(t_d) & = tt(t_d) + \beta [t^* - tt(t_d) - t_d]^+ + \gamma[tt(t_d) + t_d - t^*]^+ \\
           & = tt_a(t_a(t_d)) + \beta [t^* - t_a(t_d)]^+ + \gamma [t_a(t_d) - t^*]
  \end{split}
\end{equation}

Note now that the arrival time can be directly observed:
its dependency on the departure time \(t_d\) can thus be omitted,
and the cost directly expressed in function of the arrival time \(t_a\).

If~\eqref{eq:tt_a_def} is possible, \eqref{eq:cost_t_d_t_a} simply becomes
\begin{equation}
  \label{eq:cost_simplified}
  C(t_a; \beta, \gamma, t^*) = tt_a(t_a) + \beta [t^* - t_a]^+ + \gamma [t_a - t^*]^+
\end{equation}

Being this a relevant simplification of the problem,
it will be worth to study wether a function adhering to the specifications in \eqref{eq:tt_a_def} actually exists.

By expressing \eqref{eq:tt_a_def} in terms of the arrival time \(t_a\),
we get a definition of the function \(tt_a\):
\begin{equation*}
  tt_a(t_a) = tt(t_d(t_a))
\end{equation*}

The function is thus well defined as long as the function \(t_a(t_d)\) is invertible.
But this is always true with some reasonable assumptions:
if indeed the traffic is assumed to respect the \textit{First In First Out} principle
(according to which it is impossible that, by departing later, a commuter arrives earlier or at the same time),
the function \(t_a(t_d)\) is strictly increasing, and thus invertible.

More formally, we are assuming that, given \(t_1 > t_2\),
\begin{equation*}
  t_a(t_1) > t_a(t_2)
\end{equation*}

By applying the definition of the function \(t_a(t_d)\), we get
\begin{align*}
  t_1 + tt(t_1) & > t_2 + tt(t_2) \\
  t_1 - t_2 & > - tt(t_1) - (-tt(t_2))
\end{align*}
for each choice of \(t_1 > t_2\).

In particular, setting \(t_1 = t_2 + h\) yields
\begin{align*}
  - tt(t_2) - (- tt(t_2 + h)) & < h\\
  \frac{tt(t_2) - tt(t_2 + h)}{h} & > -1
\end{align*}
Finally, computing the limit for \(h \rightarrow 0\) yields a bound for the derivative of the travel time function:
\begin{equation}
  \label{eq:bound_der_tt}
  tt'(t) > -1
\end{equation}
that is, thus, a reasonable assumption.

The same reasoning can be repeated for the function \(tt_a\),
linking the arrival time to the travel time:
it will, in this case, yield the following bound for the derivative
\begin{equation}
  \label{eq:bound_der_tt_a}
  tt_a'(t) < 1
\end{equation}

These bounds will always be satisfied for realistic travel time functions,
and will have to be considered when defining theoretical travel time functions.

Please note that, on top of being necessary (as long as the travel time function is differentiable, that is not a restrictive assumption),
assumption \eqref{eq:bound_der_tt} is sufficient for defining the function \(tt_a\):
Computing the derivative of the function \(t_a(t_d)\) yields indeed
\begin{align*}
  t_a'(t) & = \diff{}{t}(t + tt(t)) \\
          & = 1 + tt'(t) \\
          & > 1 - 1 = 0
\end{align*}

The function is thus strictly increasing, and so invertible.

Under some weak assumptions, the function to be minimized is thus considerably simplified:
as shown in~\eqref{eq:cost_simplified}, it is now enough to minimize the function

\begin{equation}
  \label{eq:cost_simplified_final}
  C(t_a; \beta, \gamma, t^*) = tt_a(t_a) + \beta [t^* - t_a]^+ + \gamma [t_a - t^*]^+
\end{equation}

In the following pages,
the points where the minima for this function occur will be studied.

%%% Local Variables:
%%% mode: LaTeX
%%% TeX-master: "../main"
%%% End:
