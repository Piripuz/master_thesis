\missingfigure{Remove todonotes package!!}

\chapter{Introduction}
\label{cha:introduction}

\section{Background}
Urban mobility poses growing challenges for large cities, where rising travel demand interacts with limited infrastructure. Among the most pressing issues is congestion, which imposes significant delay costs on both individuals and the system as a whole. A recent empirical study in California shows that congestion leads to an average delay of four to five minutes per morning commuter per day, amounting to an annual economic loss of approximately six billion USD, equivalent to 1.67\% of the state's GDP \parencite{kim2022congestion}. These delays reflect not just inefficiencies in traffic flow but deeper structural issues in how urban mobility is managed. To address these challenges, travel demand management (TDM) has emerged as a key strategy aimed at influencing when, where, and how people travel,
rather than simply expanding infrastructure. TDM policies try to reduce peak demand, encourage shifts to more efficient modes, and make better use of existing capacity. Examples include congestion pricing, parking restrictions, high-occupancy vehicle (HOV) lanes, and incentives for carpooling or public transit use.  

TDM schemes play a critical role in addressing urban congestion by influencing when, how, and by which mode people travel.  Human mobility in cities is a complex system shaped by the interactions of three main entities: travelers, who make individualized decisions about departure time, route, and mode; governments, which regulate infrastructure through pricing, emissions rules, and access restrictions; and service providers, who manage and adapt transportation services based on demand. The objectives of these actors often conflict, and the lack of coordination among them leads to systemic inefficiencies such as congestion, underutilized capacity, and inequitable access. 

Generally, implementing effective TDM strategies requires a deep understanding of travelers' scheduling preferences and responsiveness to policy interventions. TDM effectiveness depends on accurate information about travelers' scheduling preferences-how they trade off travel time, cost, and departure flexibility. Traditionally, this information has been gathered through stated-preference (SP) surveys, which are costly, time-consuming, and often limited in scale. In contrast, revealed-preference (RP) data (such as historical traffic counts, travel times, and mode shares) are readily available and increasingly rich due to advancements in sensing and data collection technologies. This raises a key research question: Can commuters' scheduling preferences be reliably inferred from large-scale RP data without relying on direct surveys? Addressing this question is crucial for designing data-driven, adaptive TDM strategies in complex urban environments.

Transport modeling initially relied on static models, which assume that congestion levels remain constant across time. Despite this strong assumption, static models are still widely used in practice (see also \textcite{duranton2011fundamental}). This assumption was first challenged by \textcite{Vickrey1963, Vickrey1969}, who introduced the \textit{bottleneck model}, in which users trade off congestion costs against early or late arrivals relative to a preferred arrival time \( t^* \). Building on this, \textcite{arnott1993structural} extended the framework to incorporate elastic demand, simple networks, and user heterogeneity (see \textcite{Li2020} for a comprehensive synthesis).

To reflect observed variations in departure times, \textcite{de1983stochastic} introduced a \textit{stochastic departure time model}, known as the \(\alpha\)–\(\beta\)–\(\gamma\) model. In this model, \(\alpha\) denotes the value of time, while \(\beta\) and \(\gamma\) represent the penalties for early and late arrivals, respectively. The idea is simple: travelers face three types of costs. First, travel time, which we weigh using a value of time parameter,  \(\alpha\);
second, a penalty for arriving early, determined by the weight \(\beta\);
third, a penalty for arriving late, proportional to \(\gamma\).
People are assumed to choose their arrival time to minimize this total cost.
Their ideal arrival time (i. e., the start of a work shift) is denoted as $t^*$, and can vary across individuals. 

More recently, large-scale simulation tools such as METROPOLIS 1 \& 2 have incorporated departure time choice into integrated travel behavior modeling. These tools simulate users' learning processes and day-to-day adaptations until a stationary state is reached-i.e., where anticipated and experienced travel times converge throughout the day (see \textcite{javaudin2024metropolis2}).

Despite these advances, a central challenge remains: estimating the parameters \(\beta\), \(\gamma\), and \( t^* \) \textit{from RP data alone}. In large-scale networks, travel times for a given origin-destination (O-D) pair are not necessarily in equilibrium even at stationarity. This complicates the inference of user preferences based solely on observed data.

\section{Objective of the Thesis}

This thesis aims to design an effective method for inferring commuters' scheduling preferences from their observed arrival times, given an exogenous traffic congestion pattern. In our setting, the inputs are fairly intuitive: we observe when people actually arrive and the travel time they experienced. This types of data can be sourced from navigation platforms like Waze or Google Maps. Our goal is to use this information to estimate the distribution of scheduling preferences across the population, without resorting to a costly travel survey. 


The main objective is to infer travelers' scheduling preferences (characterized by $\alpha, \beta, \gamma$) and their desired arrival times $t^*$ from data, as illustrated in Figure \ref{fig:zhenyu_arrow}. 


\begin{figure}
    \centering
    \includegraphics[width=.8\linewidth]{problem.pdf}
    \caption{A schema, showing the main objectives of the thesis.
      The goal is developing a model that, given observable data about how the arrival times for a group of users are distributed,
      and the travel time they experience,
    is able to retrieve the scheduling preferences for the said group of users.}
 \label{fig:zhenyu_arrow}
\end{figure}


\section{Summary of Contributions} 


% A key contribution of this study is the development of a structural framework to infer travelers' scheduling preferences from observed arrival time patterns under an exogenously given congestion profile. We model individual decision-making as a cost-minimization problem, where each traveler chooses an arrival time by balancing congestion delays with personal penalties for arriving earlier or later than their preferred time. This framework captures heterogeneity in scheduling preferences across the population and allows us to estimate the probability distribution of arrivals throughout the day. Leveraging real-world data, we employ maximum likelihood estimation (MLE) to recover the underlying parameters governing scheduling behavior, providing a data-driven and scalable alternative to traditional survey-based methods.


% We aim to infer travelers' scheduling preferences from their observed arrival times, given an exogenous traffic congestion pattern.

In this work, we develop a methodology for calibrating the parameters for the distribution for travelers' scheduling preferences \(\beta\)–\(\gamma\) and the desired arrival time \( t^* \) in dynamic departure time models. Our approach targets specific population groups, possibly defined by socio-economic attributes and trip purposes, and uses only attainable RP data, offering a scalable and cost-efficient alternative to survey-based methods.
Our contributions can be summarized as follow:

First of all, we develop a model that explicitly characterizes the distribution of observed arrival times as a function of three key components: individual scheduling preferences (\(\beta, \gamma\)), desired arrival time (\(t^*\)), and experienced travel time as a function of arrival time (\(tt_a(t_a)\)). Each traveler selects an arrival time that minimizes their overall trip cost by considering the within-day congestion pattern along with their individual scheduling preferences. We characterize the optimal arrival time under different $\beta$, $\gamma$, and $t^*$. This framework allows us to capture the behavioral trade-offs travelers make when choosing when to arrive.

Then, by evaluating the likelihood of observed arrival time datasets, we are able to apply a maximum likelihood estimation (MLE) to recover the parameters describing the distributions of \(\beta, \gamma, t^*\) that best represent the scheduling preferences of a user group, without requiring stated-preference surveys.

Lastly, we process real data from PeMS and generate simulated data from METROPOLIS2 to gain insights regarding the real-world time-varying congestion pattern between OD pairs.  We validate our inference method on a synthetic dataset, demonstrating that it can accurately recover the wanted parameters, confirming thus the model's precision.


\section{Organization of the Thesis}
The rest of the thesis is structured in this way:
Chapter~\ref{chap:methodology} describes the methods that will be used,
explaining what methods could be used for each part of the estimation,
what method was chosen and how the estimation will be performed.
Chapter~\ref{chap:res} shows the results of the estimations:
initially, analytical results will be obtained;
once the analytical computations are done,
results regarding the numerical implementations of the developed method will be shown.
In Chapter~\ref{chap:methodology} the results presented earlier are discussed:
their strength and weaknesses are presented,
and their contributions to the current research is shown.
Lastly, Chapter~\ref{chap:conclusion} consists of a recap of the work that has been done;
finally, directions for possible future works in the field are presented.


%%% Local Variables:
%%% mode: LaTeX
%%% TeX-master: "../main"
%%% End:
