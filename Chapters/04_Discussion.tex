\chapter{Discussion}
\label{chap:discussion}

The goal of this master thesis was to develop a model that,
by uniquely relying on Revealed Preferences,
could retrieve the distribution of scheduling preferences of a group of commuters.
The development of the model has been divided in two different stages:
at first, a theoretical understanding of the problem was developed,
to be able to properly interpret the real data and,
more technically, to develop an analytical expression for the likelihood.
After this, the model was applied to different scenarios,
to evaluate its properties and to see wether it forecasts the actual scheduling delay preferences of the users.

The main contribution of this work are certainly related to the theoretical understanding of the problem:
how a rational agent minimizing the cost acts was studied,
and an analytical expression for the choices they make retrieved.
This analytical expression has moreover an intuitive geometrical meaning,
directly relating the derivative of the travel time function to the scheduling preferences of each individual:
we indeed fully characterize the period of time for which early and late arrivals are achieved,
and show how closely related they are to the derivative of the travel time function.

These results are obtained by assuming the travel time function to be exogenous,
in a situation in which the studied group of users is out from equilibrium.
This approach differs from the majority of the work in the field,
that is mainly addressed to solve various equilibrium problems.
Reversing the problem, and supposing the whole system to be at equilibrium for studying the behaviour of a part of it,
yields anyway some interesting problems.

Moreover, congestion is indeed seen, by each commuter,
as an exogenous variable:
methods that use this out of equilibrium approach should be the methods of election for inferring how commuters decide to move,
since
(while not necessarily differing much from the ones considering congestion at equilibrium)
neglecting the impact each commuter has on the traffic may simplify the studied problems,
and yield more general methods.

The validation of the method on synthetic dataset was successful.
As stated different times,
this doesn't help in proving the generalizability of the model (since it just proves that a theory is coherent with itself),
but does help in showing that the theoretical work done is relevant,
and that the numerical implementation is precise enough.

Sadly, different problems have been observed with the method.
The first one is noted in section~\ref{sec:data-about-realized}:
taking into account heteregenoities in choices,
by choosing a continuous logit model to describe the probability each departure time is chosen by an user,
the method does not work anymore,
and a deep revision of the model is made necessary.

A continuous logit model is anyway a direct specification of the Probability Density Function for the agent making the choice:
given the cost \(C(t)\),
according to it the choices are indeed assumed to be distributed following the Probability Density Function
\begin{equation*}
  f(t) = \frac{e^{C(t)/\mu}}{\int e^{C(t)/\mu}}
\end{equation*}

Implementing this method could so even simplify the estimation method,
and require less knowledge of the subject.
This would be the drawback of the method as well:
a minor understanding of the subject would be developed if implementing the logit model.

This problem can thus be overcame.
A deeper problem, that cannot be solved as easily,
would then arise when working with real data:
on real data, slopes are way lower than what would be expected if the population had parameters close to the ones reported in the literature.
This problem is reported by \textcite{https://doi.org/10.1111/iere.12692} using data from highways,
and confirmed by us when dealing with urban data.

This problem implies that, if performing an estimation on real data,
every observation will be an on-time arrival,
unless the parameters are allowed to sensibly differ from the ones reported in the literature.

Note that this does not prevent our method to be applied:
the estimation could be performed,
as long as the retrieved arrival times are consistent with what expected by Vickrey's model.
Doing an estimation with this problem would anyway not be particularly meaningful,
since the model on which the estimation is performed could itself be flawed:
a more structural approach, able to solve the problem when dealing with equilibrium situations as well, is needed to have a meaningful estimation.

Two different approaches could solve this problem:
one would leave untouched the theoretical work done in this thesis,
while the other would require a completely different approach.

The first approach consists in trying to tackle the problem from a statistical point of view:
the values reported in the literature can indeed be thought of as mean of the parameters,
allowing the parameters to be distributed more complexly across the population.
Considering a more flexible distribution,
observations of early and late arrivals can be explained by the model developed above and would,
at the same time,
not contradict the mean values reported by the literature.

This approach could explain the real data, but has some drawbacks:
it is indeed assuming a particular type of heterogeneity in the population not because the heterogeneity has been directly observed,
but rather because this type of heterogeneity would explain a problem in a theory that has not been,
historically,
frequently validated with data.

The second approach is slightly more complex, and is based on the introduction of some uncertainty in how travel time is perceived by the users:
if, instead of a deterministic travel time,
travel time is treated as a stochastic process (that is, a random variable for each time instant),
when minimizing the expected value of the cost (whose derivative has now no more discontinuities, as long as the distribution of the travel time function is behaved well enough)
the first order conditions yield
\begin{equation*}
  tt_a'(t_a) = \frac{\beta + \gamma}{\alpha}F_\varepsilon(t^* - t_a) - \frac{\gamma}{\alpha}
\end{equation*}
where \(F_\varepsilon\) is the Cumulative Distribution function for the random variable \(\varepsilon\) expressing the uncertainty in travel time.

This means that arrivals are realized, depending on how the variable \(\varepsilon\) is distributed,
for values of the derivative of the travel time lower than \(\beta\) (and higher than \(-\gamma\)).
While not being the canonical risk aversion,
this uncertainty induces a behaviour that is similar to it,
and that could explain better the real data.

Sadly, this would substantially change the model developed in these slides,
and can thus only be a direction for future research.

%%% Local Variables:
%%% mode: LaTeX
%%% TeX-master: "../main"
%%% End:
