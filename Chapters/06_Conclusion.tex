\chapter{Conclusions}
\label{chap:conclusion}

Throughout this thesis, a model for inferring how the scheduling preferences of a group of commuters are distributed using only Revealed Preference data has been developed.

The model is only a prototypical version, and can't sadly be tested on real data.
The theoretical work done in the thesis is anyway a step towards the construction of a working version of the model,
that could be deployed in real-world situations to retrieve the actual preferences of commuters.

Note that, to the best of our knowledge,
the validation of these types of bottleneck models with real-world data is a new field of research,
and work actually doing this validation is extremely rare.

Retrieving these preferences would thus be a big step in the field of traffic economics:
modelling transport shares indeed some characteristics with a market economy,
where supply and demand can clearly be identified.
A clear understanding of how the demand works, and what are the preferences that drive it,
is fundamental in order to properly forecast the response to some changes on the supply side:
by knowing the scheduling delay preferences of a group of commuters,
optimal time-varying tolls could be implemented,
or, more in general, the response to the introduction of a new policy can be precisely forecasted.

Sadly, this work was not enough to develop a fully working model.
An extensive theoretical work has anyway been done,
and results in a deep understanding of where the optimal departure times are located,
under the hypotheses customarily made in Vickrey's model.

This lays the groundwork for future research:
the model can indeed be expanded in different ways:

What achieved in section \ref{sec:minimum} can be modified to take into account other forms of uncertainty:
a continuous logit model on the departure time choice can be implemented,
or the model can be modified to consider uncertainty regarding the travel time (as suggested in chapter~\ref{chap:discussion}).

Furthermore, section~\ref{sec:lik} can be expanded as well:
some form of interdependency of the parameters \(\beta, \gamma\) can be taken into account,
and the model can be applied with more complex Probability Density Function,
as suggested in the previous chapter,
to make the method compatible with recent work that suggests a particular distribution of the parameters \parencite{https://doi.org/10.1111/iere.12692}.

The goal of future work would probably be the application of the model to real-world data,
with two potential applications.
A first application is, as said above, the implementation of better calibrated TDM policies.
Another potential application could simply be being able to study the population of commuters by simply looking at their arrival times:
the parameters \(\beta, \gamma\) could indeed be linked to some socio-economic characteristics of the individuals.
This would allow the characterization of the studied commuters by only using simple data,
such as the moment in which they leave home.

%%% Local Variables:
%%% mode: LaTeX
%%% TeX-master: "../main"
%%% End:
